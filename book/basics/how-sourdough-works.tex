In this chapter, we will cover the basics of how sourdough ferments.
First, we will look at the enzymatic reactions that take place
in your flour the moment you add water, triggering the fermentation
process. Then, in order to better understand this process, we will
learn more about the yeast and bacterial microorganisms involved.

\begin{figure}[!htb]
  \includegraphics[width=\textwidth]{infographic-enzymes}
  \caption{How amylases and proteases interact with flour}
  \label{infographic-enzymes}
\end{figure}

\section{Enzymatic reactions}

To understand the many enzymatic reactions that take place when flour
and water are mixed, we must first understand seeds and their role in
the lifecycle of wheat and other grains.

Seeds are the primary means by which many plants, including wheat,
reproduce. Each seed contains the embryo of another plant, and must
therefore contain all the nutrients that new plant requires to grow.

When the seed is dry, it is in hibernation mode and can sometimes be
stored for several years. The moment it comes into contact with water,
however, it begins to sprout. The seed turns into a germ, requiring the
stored nutrients to be converted into something the plant can use while
it grows. The catalyst that makes the associated reactions possible is water.

The seed typically contains the first prototypical leaves of the plant,
and can put down roots using the stored nutrients inside. Once those leaves
break through the soil and come into contact with the sunlight above, they
begin to photosynthesize. This process is the plant's engine, and with the
energy photosynthesis produces, the plant can continue to grow more roots,
enabling it to access additional nutrients from the soil. These additional
nutrients allow the plant to grow more leaves, increasing its photosynthetic
activity so that it can thrive in its new environment.

Of course, a ground flour can no longer sprout. But the enzymes that
trigger this process are still present. That's why it's important not to
mill grains at too high a temperature, as doing so could damage some of
these enzymes.

Normally, the grain seed shields the germ against pathogens. However, as the
grain is ground into flour, the contents of the seed are exposed. This is ideal
for our sourdough microorganisms.

% I removed the line referencing yeast as a saprotrophic fungus since you
% cover this later on in the chapter and removing that helps the text to
% flow more smoothly.
Neither the yeast nor the bacteria can prepare their own food. However, as
the enzymes are activated, the food they need becomes available, allowing them
to feed and multiply.

The two main enzymes involved in this process are \textbf{amylase} and
\textbf{protease}. For reasons that will soon be clear, they are of the utmost
importance to the home baker, and their role in the making of sourdough is a
key puzzle piece to making better-tasting bread.

\subsection{Amylase}

Sometimes, when you chew on a potato or a piece of bread for a long period
of time, you'll perceive a sweet flavor on your tongue. That's because your
salivary glands produce amylase. Amylase breaks down complex starch molecules
into easily-digestible sugars. The germ needs this to produce more plant
matter, and your body needs this to kick-start the digestive process. Normally,
the microorganisms on the surface of the grain can't consume the freed maltose
molecules, which remain hidden inside the germ. But as we grind the flour, a
feeding frenzy takes place. Generally, the warmer the temperature, the faster
this reaction occurs. That's why a long fermentation is key to making great
bread. It takes time for the amylase to break down most of the starch into
simple sugars, which are not only consumed by the yeast but are also essential
to the \textit{Maillard reaction}, responsible for enhanced browning during the
baking process.

If you're a hobby brewer, you'll know that it's important to keep your beer at
certain temperatures to allow the different amylases to convert the contained
starches into sugar \cite{beer+amylase}. This process is so important that
there's a frequently used test to determine whether or not all the starches
have been converted.

This test, called the Iodine Starch Test, involves mixing iodine into a sample
of your brew and checking the color. If it's blue or black, you know you still
have unconverted starches. I wonder if such a test would also work for bread
dough?

Industrial bakers that add especially active yeast to produce bread in a short
period of time face a similar issue. Their approach is to add malted flour to
the dough. The malted flour contains many enzymes, and thus speeds up the
fermentation process. The next time you're at the supermarket, check the
packaging of the bread you buy. If you find {\it malt} in the list of
ingredients, chances are this strategy was used.

Note that there are actually two categories of malt. One is {\it enzymatically
active malt}, which has not been heated to above 70°C, where the amylases begin
to degrade. The other is {\it inactive malt}, which has been heated to higher
temperatures and thus has no impact on your flour.

\subsection{Protease}

Just as amylase breaks starches down into simple sugars, protease breaks
complex proteins down into simpler proteins and amino acids. Because wheat
contains gluten, a protein that's essential to the structure of bread,
protease necessarily plays a crucial role in the baking of sourdough.

Since the grain seeds require smaller amino acids to build roots and other
plant materials, the gluten in those seeds will begin to break down the moment
they sprout, and since adding water to flour activates those same enzymes,
the same process occurs in bread dough.

If you've ever tried to make a wheat-based dough and kept it at room
temperature for several days, you'll have discovered for yourself that the
gluten network breaks down so that the dough can no longer hold together. Once
this happens, the dough easily tears, holds no structure, and is no
longer suitable for baking bread.

This happened to me once when I tried to make sourdough directly from a dried
starter. At three to four days, the fermentation speed was so slow that the
gluten network broke down. The root cause for this issue was protease.

By adding water to your dough, you activate the protease, and this gets to work
in readying amino acids for the germ.

Here's another interesting experiment you can try to better visualize the
importance of protease: Make a fast-proofing dough using a large quantity
of active dry yeast. In one to two hours, your dough should have leavened and
increased in size. Bake it, then examine the crumb structure. You should see
that it's quite dense and nowhere near as fluffy as it could have been. That's
because the protease enzyme wasn't given enough time to do its job.

At the start, while kneading, a dough becomes elastic and holds together very
well. As that dough ferments, however, it becomes more loose and extensible
\cite{protease+enzyme+bread}. This is because some of the gluten bonds have
been broken down naturally by the protease through a process known as
\textit{proteolysis}. This is what makes it easier for the yeast to inflate the
dough, and it's why a long fermentation process is critical when you want to
achieve a fluffy, open crumb with your sourdough bread.

Aside from using great ingredients, the slow fermentation process is one of the
main reasons Neapolitan pizza tastes so great; because the protease creates an
extensible, easy-to-inflate dough, a soft and airy edge is achieved.

Because the fermentation process typically takes longer than eight hours, a
flour with a higher gluten content should be used. This gives the dough more
time to be broken down by the protease without negatively affecting its
elasticity. If you were to use a weaker flour, you might end up with a dough
that's broken down so much that it tears during stretching, making it
impossible to shape into a pizza pie.

Traditionally, pizza has been made with sourdough, but in modern times, it is
made with active dry yeast, as the dough stays good for a longer period of time
and is much easier to handle on a commercial scale. If you were to use
sourdough, you might have a window of thirty to ninety minutes before the dough
starts to deteriorate, both because of the protease acting for a longer period
of time and the byproducts of bacteria which we'll discuss in more detail later
in this chapter.

\subsection{Improving enzymatic activity}

As explained previously, malt is a common trick used to speed up enzymatic
activity. Personally, however, I prefer to avoid malt and instead use a
trick I learned while making whole-wheat breads.

When I first started making whole-wheat bread, I could never achieve the
crust, crumb, or texture I desired no matter what I tried. Instead, my dough
tended to overferment rather quickly. When using a white flour with a similar
gluten content, however, my bread always turned out great.

At the time, I utilized an extended autolyse, which is just a fancy word for
mixing flour and water in advance and then letting the mixture sit. Most
recipes call for it as the process gives the dough an enzymatic head start, and
in general it's a great idea. However, as an equally effective alternative,
you could simply reduce the amount of leavening agent used (in the case of
sourdough, this would be your starter). This would allow the same biochemical
reactions to occur at roughly the same rate without requiring you to mix your
dough several times. My whole wheat game improved dramatically after I stopped
autolysing my doughs.

Now that I've had time to think about it, the result I observed makes sense.
In nature, the outer parts of the seed come into contact with water first, and
only after penetrating this barrier would the water slowly find its way to the
center of the grain. The seed needs to sprout first to outcompete other nearby
seeds, requiring water to enter quickly, yet the seed must also protect itself
from other animals and potentially hazardous bacteria and fungi. A way for both
goals to be accomplished would be for most of the enzymes to exist in the outer
parts of the hull. As a result, they are activated first (source needed).
Therefore, by just adding a little bit of whole flour to your dough, you should
be able to significantly improve the enzymatic activity of your dough. That's
why, for plain white flour doughs, I usually add 10\textendash20\% whole-wheat
flour.

\begin{figure}
  \includegraphics[width=\textwidth]{whole-wheat-crumb}
  \caption{A whole-wheat sourdough bread}
  \label{whole-wheat-crumb}
\end{figure}


By understanding the two key enzymes \textit{amylase} and \textit{protease}, you
will be better equipped to make bread to your liking. Do you prefer a softer
or stiffer crumb? Do you desire a lighter or darker crust? Do you wish to reduce
the amount of gluten in your final bread? These are all factors that you can
tweak just by adjusting the speed of your dough's fermentation.

\section{Yeast}

% Yeast is both the singular and plural form of the word unless you're
% specifically referencing a plural number of varieties or types, in which case
% "yeasts" would be correct.
Yeast are single celled microorganisms that belong to the fungi kingdom, and
spores that are hundreds of millions of years old have been identified by
scientists. There are a wide variety of species: So far, about 1,500 have been
identified. Unlike other members of the fungi kingdom such as mold, yeast do
not create a mycelium network \cite{molecular+mechanisms+yeast}.

\begin{figure}[!htb]
  \centering
  \includegraphics[width=1.0\textwidth]{saccharomyces-cerevisiae-microscope}
  \caption{Saccharomyces cerevisiae: Brewer's yeast under the microscope}
  \label{saccharomyces-cerevisiae-microscope}
\end{figure}

Yeast are saprotrophic fungi. This means that they do not produce their own
food, but instead rely on external sources that they can decompose and break
down into compounds that are more easily metabolized.

Yeast breaks down carbohydrates into carbon dioxide and alcohol in what we now
refer to as the fermentation process. This process has been known for thousands
of years and has been used since ancient times for the making of bread as well
as alcoholic beverages.

Yeast can grow and multiply under both aerobic and anaerobic conditions. When oxygen
is present the yeast almost completely produces
carbon dioxide and water. When no oxygen is present
the yeast starts switches its metabolism. The
yeast starts to produce alcoholic compounds \cite{effects+oxygen+yeast+growth}.
The temperatures at which the yeast grows vary. Some
yeasts such as {\it Leucosporidium frigidum} grows
best at temperatures between -2°C up to 20°C. Other
yeast grows better at higher temperatures. The warmer
it is the faster the yeast's metabolism works. The yeast
that you cultivate in your sourdough starter works best
at the temperatures where the grain was grown and at
the point when it was harvested. So if you are from a 
cooler place and cultivate a sourdough starter from
a nordic rye variety, then chances are your yeast
prefers this colder environment. As an example
beer makers discovered that a beneficial yeast lives
in the cold caves around the city of Pilsen, Czech Republic.
This yeast has produced excellent tasting beers at
lower temperatures. Varieties of these strains
are now used to make popular lager beers.

Yeasts in general are very common in the environment.
They can be found on cereal grains, fruits, other plants
in the soil and also in your gut. Very little is known
about the ecology of why yeasts we use for baking
are cultivating the leaves of the plants. The plants
are protected via the cell walls and hardly any
fungi and other bacteria can penetrate. Some fungi and
bacteria are producing enzymes that are able
to break down the cell walls and infect the plant.
There are fungi and bacteria that live within the plant
without causing any distress. These are known as {\it endophytes}.
They are not damaging the plant per se. In fact they are
living in a symbiotic relationship with the host. They
help the plant to protect itself from additional pathogens
that might enter through the leaves of the plant. They
help with water stress, heat stress and nutrient availability. 
In exchange for the service they receive carbon for energy
from the plant host. They are not always strictly mutualistic though.
Sometimes under stress conditions they can become pathogens
on their own \cite{endophytes+in+plants} and decay begin
decaying the plant.

The yeasts we use for baking are
living as as epiphytes on the plant. Compared to
the previously mentioned endophytes they are not
breaching the walls of the cells. Most of them
receive nutrients from rain water, the air or other animals.
These sources also include honeydew produced
by aphids. Pollen that lands on the leaf's surface
is an additional source of food. Interestingly
though when you remove that external food source,
you still find a large variety of epiphytic fungi
and bacteria on the plant's surface. The food
for them is coming directly from the plant it seems.
Some research has shown that the plants are
on purpose releasing some compounds such as sugars,
organic acids, amino acids, some methanol and various
salts via the surface. These nutrients would
then attract the epiphytes to live on the surface.
The plants benefit from enhanced protection against
mold and other pathogens. It is in the best interest
of the epiphytes to keep the plants alive
as long as possible \cite{leaf+surface+sugars+epiphytes}.
More and more research is conducted on using yeasts
as a biocontrol agents to protect plants. These bio-agents
would be food-safe as yeasts are generally considered save.
The yeasts would start to grow on the leaves on the plant
and essentially shield the plants from other molds. This
could be a game changer for wineyeards suffering from mildew.
This could also be helpful to shield the plant against the
psychoactive ergot fungus. The ergot fungus likes to grow
in more humid colder environments and poses a huge
problem to rye farmers. The fungus parasites the plant
and infects it. Consumption of ergot is not recommended
as it is highly toxic to the liver. That's why lawmakers
have recently reduced the amount of allowed ergot contamination
in rye flour. Another interesting experiment from Italian scientists
visualized how important yeasts could be when protecting
plants. They added tiny incisions into some of the grapes.
They would then infect some of the damaged surfaces with
mold. The other wounds they infected with some of the 150
different wild yeast strains isolated from the leaves plus
the mold. When mixing the mold with the yeast the grape
sustained no significant damage \cite{yeasts+biocontrol+agent}.
In another experiment however scientists have shown
how the brewer's yeast became an aggressive pathogen to wine plants.
Initially the yeast lived in symbiosis with the plant. After the grapevine
sustained damages the yeast became opportunistic and started to
attack the plant event producing hyphae to deeply
penetrate the plants tissue.

\section{Bacteria}