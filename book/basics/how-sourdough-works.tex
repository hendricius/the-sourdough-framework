In this chapter, we will cover the basics of how sourdough ferments.
First, we will look at the enzymatic reactions that take place
in your flour the moment you add water, triggering the fermentation
process. Then, in order to better understand this process, we will
learn more about the yeast and bacterial microorganisms involved.

\begin{figure}[!htb]
  \includegraphics[width=\textwidth]{infographic-enzymes}
  \caption{How amylases and proteases interact with flour}
  \label{infographic-enzymes}
\end{figure}

\section{Enzymatic reactions}

To understand the many enzymatic reactions that take place when flour
and water are mixed, we must first understand seeds and their role in
the lifecycle of wheat and other grains.

Seeds are the primary means by which many plants, including wheat,
reproduce. Each seed contains the embryo of another plant, and must
therefore contain all the nutrients that new plant requires to grow.

When the seed is dry, it is in hibernation mode and can sometimes be
stored for several years. The moment it comes into contact with water,
however, it begins to sprout. The seed turns into a germ, requiring the
stored nutrients to be converted into something the plant can use while
it grows. The catalyst that makes the associated reactions possible is water.

The seed typically contains the first prototypical leaves of the plant,
and can put down roots using the stored nutrients inside. Once those leaves
break through the soil and come into contact with the sunlight above, they
begin to photosynthesize. This process is the plant's engine, and with the
energy photosynthesis produces, the plant can continue to grow more roots,
enabling it to access additional nutrients from the soil. These additional
nutrients allow the plant to grow more leaves, increasing its photosynthetic
activity so that it can thrive in its new environment.

Of course, a ground flour can no longer sprout. But the enzymes that
trigger this process are still present. That's why it's important not to
mill grains at too high a temperature, as doing so could damage some of
these enzymes.

Normally, the grain seed shields the germ against pathogens. However, as the
grain is ground into flour, the contents of the seed are exposed. This is ideal
for our sourdough microorganisms.

% I removed the line referencing yeast as a saprotrophic fungus since you
% cover this later on in the chapter and removing that helps the text to
% flow more smoothly.
Neither the yeast nor the bacteria can prepare their own food. However, as
the enzymes are activated, the food they need becomes available, allowing them
to feed and multiply.

The two main enzymes involved in this process are \textit{amylase} and
\textit{protease}. For reasons that will soon be made clear, they are of the
utmost importance to the home baker and their role in the making of sourdough
is a key puzzle piece to making better-tasting bread.

\subsection{Amylase}

Sometimes, when you chew on a potato or a piece of bread for a long period
of time, you'll perceive a sweet flavor on your tongue. That's because your
salivary glands produce amylase. Amylase breaks down complex starch molecules
into easily-digestible sugars. The germ needs this to produce more plant
matter, and your body needs this to kick-start the digestive process. Normally,
the microorganisms on the surface of the grain can't consume the freed maltose
molecules, which remain hidden inside the germ. But as we grind the flour, a
feeding frenzy takes place. Generally, the warmer the temperature, the faster
this reaction occurs. That's why a long fermentation is key to making great
bread. It takes time for the amylase to break down most of the starch into
simple sugars, which are not only consumed by the yeast but are also essential
to the \textit{Maillard reaction}, responsible for enhanced browning during the
baking process.

If you're a hobby brewer, you'll know that it's important to keep your beer at
certain temperatures to allow the different amylases to convert the contained
starches into sugar \cite{beer+amylase}. This process is so important that
there's a frequently used test to determine whether or not all the starches
have been converted.

This test, called the \textit{Iodine Starch Test}, involves mixing iodine into
a sample of your brew and checking the color. If it's blue or black, you know
you still have unconverted starches. I wonder if such a test would also work
for bread dough?

Industrial bakers that add especially active yeast to produce bread in a short
period of time face a similar issue. Their approach is to add malted flour to
the dough. The malted flour contains many enzymes and thus speeds up the
fermentation process. The next time you're at the supermarket, check the
packaging of the bread you buy. If you find {\it malt} in the list of
ingredients, chances are this strategy was used.

Note that there are actually two categories of malt. One is {\it enzymatically
active malt}, which has not been heated to above 70°C, where the amylases begin
to degrade. The other is {\it inactive malt}, which has been heated to higher
temperatures and thus has no impact on your flour.

\subsection{Protease}

Just as amylase breaks starches down into simple sugars, protease breaks
complex proteins down into simpler proteins and amino acids. Because wheat
contains gluten, a protein that's essential to the structure of bread,
protease necessarily plays a crucial role in the baking of sourdough.

Since the grain seeds require smaller amino acids to build roots and other
plant materials, the gluten in those seeds will begin to break down the moment
they sprout, and since adding water to flour activates those same enzymes,
the same process occurs in bread dough.

If you've ever tried to make a wheat-based dough and kept it at room
temperature for several days, you'll have discovered for yourself that the
gluten network breaks down so that the dough can no longer hold together. Once
this happens, the dough easily tears, holds no structure, and is no
longer suitable for baking bread.

This happened to me once when I tried to make sourdough directly from a dried
starter. At three to four days, the fermentation speed was so slow that the
gluten network broke down. The root cause for this issue was protease.

By adding water to your dough, you activate the protease, and this gets to work
in readying amino acids for the germ.

Here's another interesting experiment you can try to better visualize the
importance of protease: Make a fast-proofing dough using a large quantity
of active dry yeast. In one to two hours, your dough should have leavened and
increased in size. Bake it, then examine the crumb structure. You should see
that it's quite dense and nowhere near as fluffy as it could have been. That's
because the protease enzyme wasn't given enough time to do its job.

At the start, while kneading, a dough becomes elastic and holds together very
well. As that dough ferments, however, it becomes more loose and extensible
\cite{protease+enzyme+bread}. This is because some of the gluten bonds have
been broken down naturally by the protease through a process known as
\textit{proteolysis}. This is what makes it easier for the yeast to inflate the
dough, and it's why a long fermentation process is critical when you want to
achieve a fluffy, open crumb with your sourdough bread.

Aside from using great ingredients, the slow fermentation process is one of the
main reasons Neapolitan pizza tastes so great; because the protease creates an
extensible, easy-to-inflate dough, a soft and airy edge is achieved.

Because the fermentation process typically takes longer than eight hours, a
flour with a higher gluten content should be used. This gives the dough more
time to be broken down by the protease without negatively affecting its
elasticity. If you were to use a weaker flour, you might end up with a dough
that's broken down so much that it tears during stretching, making it
impossible, for example, to shape it into a pizza pie.

Traditionally, pizza has been made with sourdough, but in modern times it is
made with active dry yeast, as the dough stays good for a longer period of time
and is much easier to handle on a commercial scale. If you were to use
sourdough, you might have a window of thirty to ninety minutes before the dough
begins to deteriorate, both because of the protease acting for a longer period
of time and the byproducts of bacteria, which we'll discuss in more detail later
in this chapter.

\subsection{Improving enzymatic activity}

As explained previously, malt is a common trick used to speed up enzymatic
activity. Personally, however, I prefer to avoid malt and instead use a
trick I learned while making whole-wheat breads.

When I first started making whole-wheat bread, I could never achieve the
crust, crumb, or texture I desired no matter what I tried. Instead, my dough
tended to overferment rather quickly. When using a white flour with a similar
gluten content, however, my bread always turned out great.

At the time, I utilized an extended autolyse, which is just a fancy word for
mixing flour and water in advance and then letting the mixture sit. Most
recipes call for it as the process gives the dough an enzymatic head start, and
in general it's a great idea. However, as an equally effective alternative,
you could simply reduce the amount of leavening agent used (in the case of
sourdough, this would be your starter). This would allow the same biochemical
reactions to occur at roughly the same rate without requiring you to mix your
dough several times. My whole wheat game improved dramatically after I stopped
autolysing my doughs.

Now that I've had time to think about it, the result I observed makes sense.
In nature, the outer parts of the seed come into contact with water first, and
only after penetrating this barrier would the water slowly find its way to the
center of the grain. The seed needs to sprout first to outcompete other nearby
seeds, requiring water to enter quickly. Yet the seed must also defend itself
against animals and potentially hazardous bacteria and fungi, requiring some
barrier to protect the embryo inside. A way for the plant to achieve both goals
would be for most of the enzymes to exist in the outer parts of the hull. As a
result, they are activated first (source needed). Therefore, by just adding a
little bit of whole flour to your dough, you should be able to significantly
improve the enzymatic activity of your dough. That's why, for plain white flour
doughs, I usually add 10\textendash20\% whole-wheat flour.

\begin{figure}
  \includegraphics[width=\textwidth]{whole-wheat-crumb}
  \caption{A whole-wheat sourdough bread}
  \label{whole-wheat-crumb}
\end{figure}


By understanding the two key enzymes \textit{amylase} and \textit{protease}, you
will be better equipped to make bread to your liking. Do you prefer a softer
or stiffer crumb? Do you desire a lighter or darker crust? Do you wish to reduce
the amount of gluten in your final bread? These are all factors that you can
tweak just by adjusting the speed of your dough's fermentation.

\section{Yeast}

% Yeast is both the singular and plural form of the word unless you're
% specifically referencing a plural number of varieties or types, in which case
% "yeasts" would be correct.
Yeast are single celled microorganisms belonging to the fungi kingdom, and
spores that are hundreds of millions of years old have been identified by
scientists. There are a wide variety of species: So far, about 1,500 have been
identified. Unlike other members of the fungi kingdom, such as mold, yeast do
not ordinarily create a mycelium network \cite{molecular+mechanisms+yeast}
\footnote{For one interesting exception, skip ahead to the end of this
section.}.

\begin{figure}[!htb]
  \centering
  \includegraphics[width=1.0\textwidth]{saccharomyces-cerevisiae-microscope}
  \caption{Saccharomyces cerevisiae: Brewer's yeast under the microscope}
  \label{saccharomyces-cerevisiae-microscope}
\end{figure}

Yeast are saprotrophic fungi. This means that they do not produce their own
food, but instead rely on external sources that they can decompose and break
down into compounds that are more easily metabolized.

Yeast breaks down carbohydrates into carbon dioxide and alcohol in what we today
refer to as the fermentation process. This process has been known for thousands
of years and has been used since ancient times for the making of bread as well
as alcoholic beverages.

Yeast can grow and multiply under both aerobic and anaerobic conditions. When
oxygen is present, they produce carbon dioxide and water almost exclusively.
When oxygen is not present, their metabolism changes to produce alcoholic
compounds \cite{effects+oxygen+yeast+growth}.

The temperatures at which yeast grows varies. Some yeasts, such as
{\it Leucosporidium frigidum}, do best at temperatures ranging from -2°C to
20°C, while others prefer higher temperatures. In general, the warmer the
environment, the faster the yeast's metabolism. The variety of yeast
that you cultivate in your sourdough starter should work best within the range
of temperatures where the grain was grown and harvested. So, if you are from a 
cooler place and cultivate a sourdough starter from a nordic rye variety,
chances are your yeast will prefer a colder environment.

As an example, beer makers discovered a beneficial yeast living in the cold
caves around the city of Pilsen, Czech Republic. This yeast has since become
known for producing excellent beers at lower temperatures and varieties of
these strains are now used for brewing popular lagers.

Yeasts in general are very common organisms. They can be found on cereal
grains, fruits, and many other plants in the soil. They can even be found
inside your gut! As it happens, the types of yeast we use for baking are
cultivated on the leaves of plants, though very little is known about the
ecology involved.

Plants are protected by thick cell walls that few fungi or bacteria can
penetrate. However, there are some species that produce enzymes capable of
breaking down those cell walls so they can infect the plant.

Some fungi and bacteria live inside plants without causing them any distress.
These are known as {\it endophytes}. Not only do they \textit{not} damage their
host, they actually live in a symbiotic relationship, helping the plants in
which they dwell to protect themselves from other pathogens that might also
come to infect them through their leaves. In addition to this protection, they
also help with water and heat stress, as well as the availability of nutrients.
In exchange for their service to their host plants, these fungi and bacteria
receive carbon for energy.

However, the relationship between endophyte and plant is not always mutually
beneficial, and sometimes, under stress, they become invasive pathogens and
ultimately cause their host to decay \cite{endophytes+in+plants}.

There are other microorganisms that, unlike endophytes, do not penetrate cell
walls but instead live on the plant's surface and receive nutrients from rain
water, the air, or other animals. Some even feed on the honeydew produced by
aphids or the pollen that lands on the surface of the leaves. Such organisms
are called \textit{epiphytes}, and included among them are the types of yeast
we use for baking.

Interestingly, when you remove external food sources, a large number of
epiphytic fungi and bacteria can still be found on the plant's surface,
suggesting that they must somehow be feeding directly from the plant.
Indeed, there is some research indicating that some plants intentionally release
compounds such as sugars, organic and amino acids, methanol, and various
salts along the surface. These nutrients would then attract the epiphytes that
live on the plant's surface.

Epiphytes are advantageous to a plant's survival, as they are provided with
enhanced protection against mold and other pathogens. Indeed, it is in the
best interest of the epiphytes to keep their host plants alive for as long as
possible \cite{leaf+surface+sugars+epiphytes}.

More research is conducted every day in ways that yeasts can be used as
biocontrol agents to protect plants, the advantage being that these bio-agents
would be food-safe as the relevant strains of yeast are generally considered
harmless to humans. The yeasts would grow and multiply on the leaves,
esentially shielding them from other types of mold. This could be a potential
game changer for vineyards that suffer from mildew.

Such bio-agents could also be used to shield plants against the psychoactive
ergot fungus, which likes to grow in colder, more humid environments and
poses a significant problem for rye farmers. Because it infects the grain
and makes it unfit for consumption due to its high toxicity to the liver,
lawmakers have recently reduced the amount of allowed ergot contamination in
rye flour.

There is another interesting experiment performed by Italian scientists that
shows how crucial yeasts could be in protecting our crops. First, they made
tiny incisions into some of the grapes on a vine. Then, they infected the
wounds with mold. Some incisions were only infected with mold. Others were also
innoculated with some of the 150 different wild yeast strains isolated from the
leaves. They found that when the wound was innoculated with yeast, the grape
sustained no significant damage \cite{yeasts+biocontrol+agent}.

Intriguingly, there was also an experiment performed that showed how brewer's
yeast could function as an aggressive pathogen to grape vines. Initially, the
yeast lived in symbiosis with the plants, but after the vines sustained heavy
damage, the yeast became opportunistic and started to attack, even going so far
as to produce hyphae, the mycellium network normally associated with a fungus,
so that they could penetrate the tissue of the plants.

\section{Bacteria}

The other more dominant microbial antagonist in your sourdough
are bacteria. They outnumber the yeast population in your sourdough
starter by 100 to 1. The bacteria is mostly responsible for creating
the sour flavour that sourdough bread is typical for. The acidity
is responsible for increasing the shelf life of sourdough breads.
\cite{shelflife+acidity}

\begin{figure}
  \includegraphics[width=1.0\textwidth]{bacteria-microscope}
  \caption{Fructilactobacillus Sanfranciscensis under the microscope}
  \label{lactobacillus-franciscensis-microscope}
\end{figure}

The bacteria in your sourdough mostly creates lactic and acetic acid. Lactic acid
has a dairy profile. Whereas the acetic acid has a more pungent
stronger vinegary profile. The bacteria are categorized into
two categories. First you have homofermentative lactic acid bacteria.
Homofermentative describes the fact that during fermentation
they mostly produce a single compound: Lactic acid. The second
category contains heterofermentative lactic acid bacteria. They
produce lactic acid, acetic acid, ethanol and even some carbon
dioxide. A quite famous strain of bacteria is called
\emph{Fructilactobacillus sanfranciscensis}. The name derives
from the famous San Francisco sourdough bread. The culture has
first been isolated from a local bakery and was then named
after the city in appreciation.

Both the yeast and bacteria compete for the same food source: sugars.
Some scientists reported how bacteria would mostly consume maltose
while the yeast consumes the glucose. Some scientists reported
how the bacteria consumes some of the compounds created by the
yeast fermentation. Similarly some of the yeast consumes left
over compounds of the bacterial fermentation. This makes sense
as nature does a very good job of composting and breaking down
everything at some point \cite{lactobacillus+sanfrancisco}.
I am still yet to find
a proper source that clearly describes the symbiosis between
the yeast and bacteria. Based on my current understanding
they both co-exist and sometimes benefit each other. The yeast
for instance tolerates the acidic environment and thus benefits
from enhanced protection from other pathogens. Other research
has shown how both the microorganisms produce compounds
to prevent the other source from consuming food. This is interesting
as it could serve as a source to identify additional antibiotics
or fungicides \cite{mold+lactic+acid+bacteria}. I have had
occasions when trying to cultivate mushrooms where you could
see the mycelium trying to defend it self from bacteria. Both
of them were actively producing compounds to combat each other.
After a while the fight between seemingly came to a standstill.
The mycelium had fully grown around the bacterial patch preventing
it from spreading any further. I imagine the same scenario happening
in a sourdough starter. As the environment tends to be more liquid
compared to when growing fungi this fight is happening in more places
at the same time, not isolated to a single patch in your dough.
More research is needed on this topic to answer the details of the
relationship between the microorganisms.

One additional trait of the bacteria is its ability to break down
and consumes proteins in your dough. If you have baked a sourdough
bread before chances are you experienced this at first hand. After
a while wheat based doughs start to break down. They seemingly become
very sticky. It becomes almost impossible to handle the dough. This
is because the bacteria starts to ferment the gluten inside of your dough.
The process is called \emph{proteolysis}. This to me was a great riddle
when starting to work with sourdough bread. Your dough becomes stickier
but at the same time it also becomes more extensible. As the gluten
is reduced it becomes easier and easier for the microorganisms to inflate the
dough. Imagine a car tire initially with thick rubber and then ultimately
a very fragile balloon. You can inflate the balloon a lot easier with your
mouth. In comparison the car tire is going to be impossible for you
to inflate. This process is further accelerated by the protease
enzyme breaking down the gluten to smaller amino acids.

This to me is the amazing process of fermentation.
When you are eating a sourdough bread you are no longer eating raw flour.
You are eating the produce of bacteria and yeast. Because of this sourdough
bread also typically
contains less gluten than a plain yeast based leavened dough
\cite{proteolysis+sourdough+bacteria}. Furthermore the bacteria
also metabolizes the ethanol produced by the yeast microorganisms and other
lactic acid bacteria. In both cases most of the resulting compounds
are organic acids. All the resources in your sourdough are recycled
as much as possible by the microorganisms. They are trying to eat whatever
is available. With each feeding they will become more adapt at using
the available resources.

Depending on which flavor you like you can adjust which organic acids
you would like your sourdough to produce. Production of acetic acid
requires the presence of oxygen. By depriving your sourdough starter
of oxygen you boosting homofermentative lactic acid bacteria in your
starter. Over time they will become dominant and outcompete the acetic acid
producing bacteria \cite{acetic+acid+oxygen}. The optimal fermentation temperature of your
lactic acid bacteria depends on the cultured strains. Generally the bacteria
work best at the same temperature used to initially setup your sourdough
starter. This has been the optimal temperature at which your strains
were set up. In another experiment scientists analyzed lactic acid bacteria
on corn leaves. They on purpose lowered the temperature from 20-25°C to around 5-10°C.
They were able to observe lactic acid bacteria that they had never seen
before \cite{temperature+bacteria+corn}. This confirms that there is a
large variety of different bacteria
strains living on the leaves of the plant. You could probably reproduce
that experiment if you started a sourdough starter at lower temperature.
Your starter's microbiome would be more adapt to fermenting at lower
temperatures. The microorganisms that best thrive at the lower temperatures
will start to become dominant. It would be an interesting experiment
to see if this could actively influence the taste of the sourdough
bread.
