In this chapter, we will cover the basics of how sourdough ferments.
First, we will look at the enzymatic reactions that take place
in your flour the moment you add water, triggering the fermentation
process. Then, in order to better understand this process, we will
learn more about the yeast and bacterial microorganisms involved.

\begin{figure}[!htb]
  \includegraphics[width=\textwidth]{infographic-enzymes}
  \caption{How amylases and proteases interact with flour}
  \label{infographic-enzymes}
\end{figure}

\section{Enzymatic reactions}

To understand the many enzymatic reactions that take place when flour
and water are mixed, we must first understand seeds and their role in
the lifecycle of wheat and other grains.

Seeds are the primary means by which many plants, including wheat,
reproduce. Each seed contains the embryo of another plant, and must
therefore contain all the nutrients that new plant requires to grow.

When the seed is dry, it is in hibernation mode and can sometimes be
stored for several years. The moment it comes into contact with water,
however, it begins to sprout. The seed turns into a germ, requiring the
stored nutrients to be converted into something the plant can use while
it grows. The catalyst that makes the associated reactions possible is water.

The seed typically contains the first prototypical leaves of the plant,
and can put down roots using the stored nutrients inside. Once those leaves
break through the soil and come into contact with the sunlight above, they
begin to photosynthesize. This process is the plant's engine, and with the
energy photosynthesis produces, the plant can continue to grow more roots,
enabling it to access additional nutrients from the soil. These additional
nutrients allow the plant to grow more leaves, increasing its photosynthetic
activity so that it can thrive in its new environment.

Of course, a ground flour can no longer sprout. But the enzymes that
trigger this process are still present. That's why it's important not to
mill grains at too high a temperature, as doing so could damage some of
these enzymes.

Normally, the grain seed shields the germ against pathogens. However, as the
grain is ground into flour, the contents of the seed are exposed. This is ideal
for our sourdough microorganisms.

Neither the yeast, considered a saprotrophic fungus, nor the bacteria can
prepare their own food. However, as the enzymes are activated, the food they
need becomes available, allowing them to feed and multiply.

The two main enzymes involved in this process are \textbf{amylase} and
\textbf{protease}. For reasons that will soon be clear, they are of the utmost
importance to the home baker, and their role in the making of sourdough is a
key puzzle piece to making better-tasting bread.

\subsection{Amylase}

Sometimes, when you chew on a potato or a piece of bread
for a long period of time, you'll begin to notice a sweet flavor in
your mouth. That's because your salivary glands produce amylase.
Amylase breaks down complex starch molecules into easily-digestible
sugars. The germ needs this to produce more plant matter, and your body
needs this to kick-start the digestive process. Normally, the microorganisms
on the surface of the grain can't consume the freed maltose molecules,
which remain hidden inside the germ. But as we grind the flour, a feeding
frenzy begins. Generally, the warmer the temperature, the faster this
reaction takes place. That's why a long fermentation is key to making
great bread. It takes time for the amylase to break down most of
the starch into simple sugars, which are not only consumed by the
yeast but are also essential to the \it{Maillard reaction}
that's responsible for enhanced browning during the baking process.

If you're a hobby brewer, you'll know that it's important to keep
your beer at certain temperatures to allow the different amylases to
convert the contained starches into sugar \cite{beer+amylase}. This
process is so important that there's a frequently used test to
determine whether or not all the starches have been converted.

This test, called the Iodine Starch Test, involves mixing iodine into
a sample of your brew and checking the color. If it's blue or black,
you know you still have unconverted starches. I wonder if such a test
would also work for bread dough?

Industrial bakers that add especially active yeast to produce bread
in a short amount of time face a similar issue. Their approach is to
add malted flour to the dough. The malted flour contains many enzymes,
and thus speeds up the fermentation process. The next time you're at the
supermarket, check the packaging of the bread you buy. If you find
{\it malt} in the list of ingredients, chances are this strategy was
used.

Note that there are actually two categories of malt. One is
{\it enzymatically active malt}, which has not been heated to above 70°C,
where the amylases start to degrade. The other is {\it inactive malt},
which has been heated to higher temperatures and thus has no impact on
your flour.

\subsection{Protease}

The second very important enzyme is the protease. Proteases
break down proteins into smaller proteins or amino acids.
Gluten for instance is a storage protein built by wheat.
The gluten is broken down and converted the moment the
seed starts to sprout. That's because the seed needs
smaller amino acids to build the roots and other plant material.
If you ever try to make a wheat based dough and just keep
it for several days at room temperature you will notice
how your gluten network starts to break down. The dough
no longer holds together. You can just fully tear it apart.
I have had this happen to me when I was trying to make
doughs directly with dried sourdough starter. The fermentation
speed was so low that it took 3-4 days for the dough
to be ready. The root cause for this issue is the protease.
By adding water to the dough the protease was activated
and started to ready amino acids for the germ in order to be
able to sprout. Another interesting experiment that viusalises
the importance of protease is the following. Try to make a
fast dough within 1-2 hours. Simply use a large quantity
of dry yeast. Your dough will be leavened and increase in size.
Bake your dough and notice the crumb of your baked dough.
You will notice that the crumb is quite dense and not as
fluffy as it could be. That's because the protease enzyme
didn't have enough time to do its job. At the start
when kneading your dough is very elastic. It holds together
very well. Over the course of the fermentation process
your dough will become more extensible \cite{protease+enzyme+bread}.
Some of the gluten bonds start to naturally break
down due to the protease proteolysis. This makes it easier
for your dough to be inflated. That's why a long
fermentation process is important when you want to
achieve very fluffy and open crumbs with your sourdough
bread. Next to using great ingredients, the long and
slow fermentation is one of the main reasons why
Neapolitan pizza tastes so great. The soft and fluffy
edge of the pizza is achieved because of the protease
creating a very extensible easy to inflate dough. Because
the fermentation process is typically longer than 8
hours a flour with a higher gluten content is used. There
is more gluten that can be broken down by the protease.
By using a weaker flour you might end up with a dough
that's already broken down too much and will then tear
when trying to make a pizza pie. Traditionally the pizza
has probably been made with sourdough. In modern times
it is made with yeast as handling a yeast based
dough can be done easier on a larger scale. The dough
stays good for a longer period of time. If you were to use
sourdough you might have a window of 30-90 minutes when
your dough is perfect. Afterwards the dough might
start to deteriorate because of bacteria breaking
down the gluten network too much.

\subsection{Improving enzymatic activity}

As explained previously malt is a common trick used
to speed up enzymatic activity. I personally prefer
to avoid malt in most of my recipes. Instead I use
a trick I observed when making whole wheat doughs.
No matter what I tried I could never achieve baking
a whole wheat bread with the desired crust and crumb
texture I was looking for. My doughs would tend to
overferment relatively quickly. When using a flower
with a similar amount of gluten that didn't contain
bran and other outer parts of the grain my doughs turned
out great. I was utilizing an extended autolyse.
That's a fancy word for just mixing flour and water in
advance and letting that mixture sit. Most recipes
call for it as the help to make a dough that has already
started to break down by enzymes. In general it's a great
idea but at the same time you can just reduce the amount
of leavening agent you use. This way the same biochemical
reactions happen and you don't have to mix your dough
several times. My whole wheat game drastically improved
when I stopped using the autolysis. It makes sense if I
think about it now. The first parts of the seed that
are in contact with water are the outer parts. Water
will slowly enter the center parts of the grain. The
moment the seed starts to sprout it needs to outcompete
other nearby seeds. Furthermore it also directly becomes
exposed to other animals and potential hazardous bacteria
and fungi. To accelerate this process most of the enzymes
of the grain are in the outer parts of the hull. They
are being activated first (source needed). So by just
adding a little bit of whole flour to your dough you 
will improve enzymatic activity of your dough. That's
why most of my plain flour doughs typically contain
at least 10-20 percent whole wheat flour.

\begin{figure}
  \includegraphics[width=\textwidth]{whole-wheat-crumb}
  \caption{A whole wheat sourdough bread}
  \label{whole-wheat-crumb}
\end{figure}


By understanding the 2 key enzymes amylase and protease
you will better be able to understand how to make a
dough to your liking. Would you like a dough a softer
or stiffer crumb? Would you like to achieve a darker crust?
Would you like to reduce the amount of gluten in your
final bread? These are all factors you can influence
by adjusting the speed of fermentation.

\section{Yeast}

Yeasts are single celled microorganisms that are part of
the fungus kingdom. Yeast spores that are hundreds
of million years old have been identified by scientists.
There is a wide variety of species and so far around 1500
different species have been recognized. Yeasts are not creating
a mycelium network like mold does for instance
\cite{molecular+mechanisms+yeast}.

\begin{figure}[!htb]
  \centering
  \includegraphics[width=1.0\textwidth]{saccharomyces-cerevisiae-microscope}
  \caption{Saccharomyces cerevisiae: Brewer's yeast under the microscope}
  \label{saccharomyces-cerevisiae-microscope}
\end{figure}


Yeasts are saprotrophic fungi. This means they are not
producing their own food. They rely on external food sources
which they decompose and break down. For yeasts
carbohydrates and broken down to carbon dioxide and
alcohols. The products of this fermentation process
have been used for thousands of years when making
bread or alcoholic beverages. Yeasts can grow
in both aerobic and anaerobic conditions. When oxygen
is present the yeast almost completely produces
carbon dioxide and water. When no oxygen is present
the yeast starts switches its metabolism. The
yeast starts to produce alcoholic compounds \cite{effects+oxygen+yeast+growth}.
The temperatures at which the yeast grows vary. Some
yeasts such as {\it Leucosporidium frigidum} grows
best at temperatures between -2°C up to 20°C. Other
yeast grows better at higher temperatures. The warmer
it is the faster the yeast's metabolism works. The yeast
that you cultivate in your sourdough starter works best
at the temperatures where the grain was grown and at
the point when it was harvested. So if you are from a 
cooler place and cultivate a sourdough starter from
a nordic rye variety, then chances are your yeast
prefers this colder environment. As an example
beer makers discovered that a beneficial yeast lives
in the cold caves around the city of Pilsen, Czech Republic.
This yeast has produced excellent tasting beers at
lower temperatures. Varieties of these strains
are now used to make popular lager beers.

Yeasts in general are very common in the environment.
They can be found on cereal grains, fruits, other plants
in the soil and also in your gut. Very little is known
about the ecology of why yeasts we use for baking
are cultivating the leaves of the plants. The plants
are protected via the cell walls and hardly any
fungi and other bacteria can penetrate. Some fungi and
bacteria are producing enzymes that are able
to break down the cell walls and infect the plant.
There are fungi and bacteria that live within the plant
without causing any distress. These are known as {\it endophytes}.
They are not damaging the plant per se. In fact they are
living in a symbiotic relationship with the host. They
help the plant to protect itself from additional pathogens
that might enter through the leaves of the plant. They
help with water stress, heat stress and nutrient availability. 
In exchange for the service they receive carbon for energy
from the plant host. They are not always strictly mutualistic though.
Sometimes under stress conditions they can become pathogens
on their own \cite{endophytes+in+plants} and decay begin
decaying the plant.

The yeasts we use for baking are
living as as epiphytes on the plant. Compared to
the previously mentioned endophytes they are not
breaching the walls of the cells. Most of them
receive nutrients from rain water, the air or other animals.
These sources also include honeydew produced
by aphids. Pollen that lands on the leaf's surface
is an additional source of food. Interestingly
though when you remove that external food source,
you still find a large variety of epiphytic fungi
and bacteria on the plant's surface. The food
for them is coming directly from the plant it seems.
Some research has shown that the plants are
on purpose releasing some compounds such as sugars,
organic acids, amino acids, some methanol and various
salts via the surface. These nutrients would
then attract the epiphytes to live on the surface.
The plants benefit from enhanced protection against
mold and other pathogens. It is in the best interest
of the epiphytes to keep the plants alive
as long as possible \cite{leaf+surface+sugars+epiphytes}.
More and more research is conducted on using yeasts
as a biocontrol agents to protect plants. These bio-agents
would be food-safe as yeasts are generally considered save.
The yeasts would start to grow on the leaves on the plant
and essentially shield the plants from other molds. This
could be a game changer for wineyeards suffering from mildew.
This could also be helpful to shield the plant against the
psychoactive ergot fungus. The ergot fungus likes to grow
in more humid colder environments and poses a huge
problem to rye farmers. The fungus parasites the plant
and infects it. Consumption of ergot is not recommended
as it is highly toxic to the liver. That's why lawmakers
have recently reduced the amount of allowed ergot contamination
in rye flour. Another interesting experiment from Italian scientists
visualized how important yeasts could be when protecting
plants. They added tiny incisions into some of the grapes.
They would then infect some of the damaged surfaces with
mold. The other wounds they infected with some of the 150
different wild yeast strains isolated from the leaves plus
the mold. When mixing the mold with the yeast the grape
sustained no significant damage \cite{yeasts+biocontrol+agent}.
In another experiment however scientists have shown
how the brewer's yeast became an aggressive pathogen to wine plants.
Initially the yeast lived in symbiosis with the plant. After the grapevine
sustained damages the yeast became opportunistic and started to
attack the plant event producing hyphae to deeply
penetrate the plants tissue.

\section{Bacteria}