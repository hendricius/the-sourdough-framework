\documentclass[a4paper, 12pt]{book}
\usepackage[utf8]{inputenc}
\usepackage{blindtext}

\usepackage{graphicx}
\graphicspath{ {./images/} }


\title{%
  The Sourdough Framework\\
  \large How to master making bread at home\\
  \small - the bread code community book -}

\author{Hendrik Kleinwächter}
\date{\today}

\begin{document}

\begin{titlepage}
\maketitle
\end{titlepage}


\frontmatter

\tableofcontents

\chapter{Foreword}

Still need someone to write a foreword

\chapter{Preface}

If there is the one food from Germany it is probably bread. There are thousands
of different varieties of bread in Germany. Making bread has been an integral part
of our culture. Beginning my studies in Göttingen for the first time I was faced
with buying bread on my own. In Germany that is no easy task
as the varieties of bread are endless. I started to check the packaging
of different bread types and noticed how there were surprisingly
many ingredients in most of the breads found in a common supermarket.


\chapter{Acknowledgements}

This book would not have been possible without the help of the community.


\mainmatter

\chapter{The history of sourdough}
\section{Sourdough bread in ancient times}
\section{How modern bread is made}
\section{Sourdough in modern times}

\chapter{How sourdough works}
\section{Enzymatic reactions}
\section{Yeast}
\section{Lactic acid bacteria}
\section{Acetic acid bacteria}

\chapter{Making a sourdough starter}
\section{Baker's math}
\section{The process of making a starter}
\section{How flour is fermented}
\section{Determining starter readiness}
\section{Maintenance}
\section{Longterm starter storage}

\chapter{Sourdough starter types}
\section{The regular starter}
\section{Stiff starter}
\section{Liquid starter}
\section{Lievito madre}

\chapter{Flour types}
\section{Wheat like}
\section{Non gluten binding}
\section{Gluten free}
\section{Blending flours}

\chapter{Bread types}
\section{Wheat bread basics}
\section{Non wheat bread basics}
\section{The simplest way to make bread}

\chapter{Wheat sourdough}
\section{The process}
\section{Readying your starter}
\section{Ingredients}
\section{Hydration}
\section{Autolyse}
\section{Fermentolyse}
\section{Dough strength}
\section{Controlling fermentation}
\section{Optional Preshaping}
\section{Shaping}
\section{Proofing}

\chapter{Non wheat bread basics}
\section{Ingredients}
\section{Managing acidity}
\section{To shape or not to shape}
\section{Proofing}

\chapter{Baking}
\section{The role of steam}
\section{Temperature}
\section{Home oven setup}
\section{Dutch ovens}

\chapter{Storing bread}
\section{Fridge}
\section{Room temperature}
\section{Frozen}

\chapter{Troubleshooting}
\section{Debugging your crumb structure}
\section{Baking in the tropics}
\section{My bread stays flat}
\section{I want more tang in my bread}
\section{My bread is too sour}
\section{Fixing a moldy sourdough starter}
\section{My bread flattens out after shaping}
\section{Liquid on top of my starter}
\section{Why does my starter smell like acetone}

\end{document}