\begin{quoting}
This glossary provides definitions and explanations for terms frequently
used in bread making. Understanding these terms is essential for both
novice and experienced bakers aiming to master the art and science of
bread making. The glossary is arranged alphabetically for easy reference.
\end{quoting}

\textbf{Acetic Acid:} A type of organic acid produced by hetero fermentative lactic
acid bacteria and acetic acid bacteria during fermentation. It gives sourdough bread
its characteristic tangy flavor and helps to preserve the bread by lowering its pH.
The flavor of acetic acid has a more vinegary profile.

\textbf{Alveograph:} A device used primarily in the evaluation of wheat flour's
baking quality. The alveograph assesses the dough's rheological properties,
particularly its extensibility and resistance to extension, by inflating a piece
of dough like a balloon until it bursts. The resulting chart, or ``alveogram'',
displays a curve that represents the balance between the dough's elasticity and
extensibility. Specific parameters derived from the curve, such as the P (pressure
required to inflate the dough) and L (extensibility of the dough), provide invaluable
insights to bakers and millers regarding the flour's potential performance in
bread-making. By analyzing the alveogram, professionals can make informed decisions
about the suitability of a flour for certain baking applications, as well as
potential blending needs with other flours.

\textbf{Alveoli (sg: Alveolus):} The little pockets that form the crumb, formed by
the gluten matrix trapping carbon dioxide.

\textbf{Alpha-amylase:} A type of amylase that breaks down starch molecules into
shorter fragments, producing maltose and some glucose.

\textbf{All purpose flour:} A general flour that’s balanced to make breads and also
cakes. In germany this is type 550.

\textbf{Amylase:} An enzyme that breaks down starches into simpler sugars, facilitating
the fermentation process in beer and bread making. When making beer the temperature
of the brew is kept for extended periods at certain temperatures to ensure that most
starches are broken down to sugars. These sugars are then consumed by the microbes
during the fermentation process.

\textbf{Autolyse:} A process where flour and water are mixed and then left to rest
before adding other ingredients. This activates enzymes such as amylase and protease.
By doing so the bulk fermentation time is shortened and the final loaf will have
better properties. The browning of the loaf becomes better and the crumb fluffier.
An autolyse is recommended when using a high percentage of starter to inoculate the
dough (> 20\%). An alternative easier approach can be the fermentolyse.

\textbf{Baker’s math:} Baker’s math is a ratio based system of sharing recipes,
making them easily scalable. It’s based on the total weight of the flour in a formula,
where each ingredients weight is divided by the flours weight to give a percentage.
For 500 grams of flour you could be using 60\% of water (300 grams), 10\% of starter
(50 grams) and 2\% of salt (10 grams).

\textbf{Baker’s percentage:} See Baker’s math.

\textbf{Baking:} The final, transformative step in bread making wherein dough is
exposed to high temperatures, causing a series of chemical and physical reactions
that result in a finished loaf of bread. During the baking stage:

1: \textit{Yeast Activity \& Oven Spring:} In the initial phase of baking, the
temperature inside the dough rises, increasing yeast activity. This results in rapid
carbon dioxide production, leading to what bakers refer to as ``oven spring'', or the
rapid rise of the loaf.

2: \textit{Protein Coagulation:} As the temperature continues to climb, the proteins
in the dough, primarily gluten, begin to coagulate or set, which gives the bread its
structure.

3: \textit{Starch Gelatinization:} Starches absorb water and swell, eventually
gelatinizing. This process contributes to the crumb structure of the bread.

4: \textit{Caramelization \& Maillard Reaction:} The crust of the bread browns due
to two primary reactions: caramelization of sugars and the Maillard reaction between
amino acids and reducing sugars. This not only affects the appearance but also imparts
a distinctive flavor and aroma to the bread.

5: \textit{Evaporation of Acids:} Some acids produced during fermentation evaporate at
certain temperatures during baking. This evaporation can influence the final flavor
profile of the bread, making it less tangy than the unbaked dough. By extending the
baking time the acids become less concentrated and the dough can lose some of its tang.

6: \textit{Moisture Evaporation:} Water in the dough turns to steam and begins to
evaporate. The steam contributes to the oven spring and also helps in gelatinizing
the starches.

7: \textit{Crust Formation:} The outer layer of the dough dries out and hardens to
form a crust, which acts as a protective barrier, keeping the inner crumb moist.

\textbf{Banneton:} A wicker basket used to shape and support dough during its final
proof. The bannetons are typically made out of rattan or wood pulp. An alternative
DIY solution is to use a bowl with a kitchen towel inside. While resting inside of
the banneton the dough’s surface dries out and becomes easier to score before baking.

\textbf{Bench rest:} A short resting period given to the dough after preshaping
allowing the gluten to relax a little bit and making shaping easier. Most people
bench rest for 10 minutes up to an hour. The bench rest becomes especially important
when making pizza doughs. Without an extended bench rest the dough is too elastic and
can not be shaped.

\textbf{Beta-amylase:} An enzyme that further breaks down the starch fragments
produced by alpha-amylase into maltose.

\textbf{Bread flour:} A flour that is perfect for sourdough bread making. It features
a higher amount of gluten and can thus ferment for a longer period of time.

\textbf{Brühstück:} A German baking technique similar to a scald. It translates as
``boil piece''. Hot or boiling water is poured over whole grain flour or crushed grains,
then cooled and mixed with the main dough. This process helps in moisture retention
and can enhance the flavor and texture of the final bread. Also see \#Scald.

\textbf{Bulk Fermentation:} The initial rising period after mixing all the ingredients.
The dough is typically allowed to rise until it increases to a certain volume. The
volume of increase depends on the flour that is used. When baking with wheat flour
the gluten amount of the flour is the deciding factor. The more gluten your flour has
(protein) the longer you can bulk ferment. A longer bulk fermentation improves the
flavor and texture of the final bread. It becomes tangier and fluffier. You can aim
for a 25\% size increase of your dough and then slowly increase this to find your
flour’s sweetspot. This is highly dependant from flour to flour. When using low gluten
flour like rye you need to be careful as the longer fermentation can create a too
sticky dough which collapses and does not hold its shape anymore.

\textbf{Cake flour:} Cake flour is a light, finely milled flour with a lower protein
content than all-purpose flour. It's ideal for tender baked goods like cakes, cookies,
and pastries.

\textbf{Scald:} A method where boiling water is poured over flour, grains, or other
ingredients and then allowed to cool. In baking, this process can gelatinize the
starches in the flour or grains, resulting in a dough that retains moisture better,
provides a softer crumb, and potentially extends the bread's shelf life. Additionally,
scalding can help inactivate certain enzymes which can be detrimental to the dough's
quality. The scalding technique can also enhance the overall flavor and aroma of
the bread, bringing out more pronounced grainy notes and reducing bitterness
sometimes found in certain whole grains.

\textbf{Coil fold:} A special stretch and folding technique. The coil fold is
very gentle on the dough and is thus excellent throughout the bulk fermentation.
By applying the coil fold the dough strength is improved by minimising damage
to the dough structure.

\textbf{Crumb:} The inner texture of the bread, which is characterized by the size,
shape, and distribution of the holes (or ``alveoli''). It's what's inside once you slice
a loaf of bread open. A ``tight crumb'' refers to bread with small, evenly distributed
holes, while an ``open crumb'' has larger, more irregular holes.

\textbf{Diastatic malt:} Malted grain that has been dried and then ground into a powder.
This malt contains enzymes that can break down starches into sugars, which can be
beneficial in the fermentation process for bread. When added to dough, it can improve
the bread's flavor, color, and shelf life.

\textbf{Discard:} The portion of sourdough starter that is removed and not fed when
maintaining the starter. This is often done to prevent the starter from becoming too
large and unmanageable. Discard can be used in various recipes or thrown away.

\textbf{Dividing:} The process of breaking the dough mass into smaller pieces,
typically to shape into individual loaves or portions.

\textbf{Dough Hydration:} Expressed as a percentage, it's the amount of water in a
dough relative to the amount of flour. A higher hydration dough will be wetter and
stickier, while a lower hydration dough will be firmer. For example, a dough with 500g
of flour and 375g of water has a hydration of 75\%.

\textbf{Dough Strength:} Refers to the dough's resilience, elasticity, and structure.
A strong dough can be stretched without tearing and holds its shape well. This is
largely influenced by the flour's protein content and the development of the gluten
network.

\textbf{Dutch oven:} A heavy-duty pot with a tight-fitting lid, often made of cast
iron. It's used in baking to trap steam during the initial phase of baking, helping
to create a crusty exterior on bread.

\textbf{Feed:} The act of adding fresh flour and water to maintain a sourdough
starter. Regular feeding keeps the starter active and healthy.

\textbf{Fermentation:} The metabolic process by which microorganisms such as yeast
and bacteria convert carbohydrates (like sugars) into alcohol or acids. In bread
making, this produces carbon dioxide which causes the dough to rise.

\textbf{Fermentolyse:} Using a small amount of starter to slow fermentation.
It's a method where fermentation and autolyse are combined. Typically around 10\%
of starter is used for the fermentolyse. The flour, water and starter are mixed
together. By adding the starter early the dough becomes more extensible and easier
to handle.

\textbf{Fool’s Crumb:} A term used to describe a crumb structure that has several
large pockets or holes, rather than an even distribution of smaller holes. This
isn't necessarily a desired feature, as it can indicate uneven fermentation or
improper shaping techniques.

\textbf{Elasticity:} A property of dough that describes its ability to return to
its original shape after being stretched or deformed. It's influenced by the flour's
protein content and the development of the gluten network.

\textbf{Extensibility:} Refers to the dough’s ability to be stretched or extended
without tearing. It's the opposite of elasticity and is desirable in certain types
of breads, like ciabatta, that have a more open crumb structure.

\textbf{Homogenizing:} The act of creating a consistent and uniform mixture. For
flours like einkorn and rye, where gluten alignment isn't the main goal, kneading
ensures that the dough achieves this homogeneous consistency.

\textbf{Gluten:} A protein complex formed from gliadin and glutenin, found in wheat
and some other grains. It provides elasticity and strength to the dough when
properly aligned and developed. During the course of the bulk fermentation much of
the gluten is degraded by the protease enzyme and lactic acid bacteria.

\textbf{Hooch:} A liquid layer that sometimes forms on top of a sourdough starter.
It's an indication that the starter is hungry and needs feeding. It acts as a
barrier shield and prevents the starter from catching mold. It can be mixed right
back into the starter or extracted to make hot sauces.

\textbf{Kneading:} The manual or mechanical process of working dough to develop gluten
in wheat and spelt-based breads, or to homogenize the dough mass in flours like
einkorn or rye.

\textbf{Kochstück:} When making a Kochstück, the flour or grains are heated
together with the fluid. The mixture needs to be stirred while heating up
to prevent clumping and burning it.

\textbf{Lactic Acid:} Another organic acid produced by lactic acid bacteria during
fermentation. It imparts a mild tangy yogurty flavor to sourdough bread and, along
with acetic acid, contributes to the bread's overall acidity.

\textbf{Maillard Reaction:} The Maillard reaction is one of the causes of food browning
during cooking. The reaction occurs between reducing sugars and amino acids, and
depending on the initial reactants and cooking conditions can produce a wide variety
of end products with different tastes and aromas. Maillard reactions occur readily
above 150 degrees centigrade, although will still occur much more slowly below that
temperature. Optimal reaction rate occurs between pH 6 to 8, although it favours
alkaline conditions.

\textbf{Maltose:} A sugar produced from the enzymatic breakdown of starch by amylases.
It's a primary food source for yeast during fermentation.

\textbf{Non-diastatic malt:} Malted grain that has been dried at higher temperatures,
deactivating its enzymes. It's used primarily for flavor and color in bread making.
Amylase and protease become degraded at temperatures higher than 50°C.

\textbf{Over fermenting:} A common problem when making wheat or spelt doughs. When the
dough is fermented for too long most of the gluten in the dough is broken down. The
resulting dough is very sticky. The final bread will be very flat and lose some of its
typical texture. The crumb structure features many tiny pockets of air. A lot of the
trapped gasses can diffuse out of the dough during baking. If you notice this during
bulk fermentation it is advised to place the loaf inside of a loaf pan and then bake
it after a 30 to 60 minute rest.

\textbf{Oven Spring:} The rapid rise of the dough in the oven during the early stages
of baking due to the expansion of trapped gases and water.

\textbf{Over proofing:} The same as over fermenting, however happening during the
proofing stage.

\textbf{pH:} A measure of the acidity or alkalinity of a solution. The pH scale
ranges from 0 to 14, where a pH value of 7 is neutral. Solutions with a pH value below
7 are acidic, while those with a pH above 7 are alkaline or basic. Fermented
foods with a pH below 4.2 are generally considered foodsafe. A pH meter can be
used to monitor your sourdough bread's fermentation progres.

\textbf{P/L Value:} A critical parameter derived from the alveograph test, the P/L
value represents the ratio of the dough's tenacity (P) to its extensibility (L).
Specifically:
\begin{itemize}
    \item \textbf{P (Pressure):} Refers to the pressure required to inflate the dough
    during the alveograph test. It indicates the dough's resistance to deformation or
    its strength.
    \item \textbf{L (Length):} Represents the extensibility of the dough, or how far it
    can be stretched before tearing.
\end{itemize}
The P/L ratio provides insights into the balance between the dough's elasticity and
extensibility:
\begin{itemize}
    \item \textbf{Low P/L Value:} Indicates a dough that is more extensible than
    resistant. This means the dough can be stretched easily, making it suitable for
    certain products like pizza or ciabatta.
    \item \textbf{High P/L Value:} Suggests a dough that has more strength than
    extensibility. Such a dough is more resistant to deformation, which can be
    preferable for products that require good volume and structure, like certain types
    of bread.
\end{itemize}
The P/L value helps bakers and millers determine the suitability of a flour for
specific baking applications. Adjustments in flour blends or baking processes might
be made based on this ratio to achieve desired bread characteristics.

\textbf{Preferment:} A mixture of a proportion of the doughs ingredients which is
allowed to ferment before being added to the final bread dough. These can include
sourdough, poolish, biga, pâte fermentée, or a general sponge.

\textbf{Preshaping:} When dividing your large dough mass into smaller portions you end
up having non-uniform pieces of dough. This makes shaping much harder because the
resulting shaped dough will not be uniform. For this reason bakers drag the tiny dough
pieces over the surface of the counter to create more uniform looking dough balls.

\textbf{Proof:} The final rise of the shaped dough before baking.

\textbf{Protease:} An enzyme that breaks down proteins, including gluten, into smaller
peptide chains and amino acids. In the context of bread making, protease activity can
both benefit and challenge bakers. Moderate protease activity can make dough more
extensible, which can be helpful in some bread-making processes. However, excessive
protease activity can weaken the gluten network, leading to doughs that are slack,
sticky, and challenging to handle, and may result in breads with poor volume and
structure. Factors such as fermentation time, dough temperature, and the source of the
flour can influence protease activity in bread doughs. In sourdoughs, longer
fermentation times, particularly at warmer temperatures, can lead to higher protease
activity, as the acidic conditions activate cereal proteases. Flour from sprouted
grains or malted grains can have higher protease activity due to the sprouting or
malting process. Understanding and controlling protease activity is crucial in
achieving desired bread quality and handling characteristics.

\textbf{Pullman Loaf:} A type of bread loaf characterized by its perfectly rectangular
shape and soft, fine crumb. It is baked in a special lidded pan called a Pullman pan
or pain de mie pan. The lid ensures that the bread rises in a perfectly straight
shape, without the domed top characteristic of other bread loaves. Pullman loaves are
often sliced very thin and are popular for making sandwiches.

\textbf{Rye:} A type of grain used in baking. Due to its low gluten content, breads
made solely from rye flour tend to be dense. However, rye has a unique flavor and
many health benefits, so it's often combined with wheat flour in baking. Pure rye
breads are typically made with a sourdough process to help the dough rise.

\textbf{Straight Dough:} A bread-making method where all ingredients are mixed
together at once, without the use of a preferment.

\textbf{Stretch and Fold:} S\&F is a technique used during the bulk fermentation phase
to strengthen the dough and help align the gluten structure. Instead of traditional
kneading, the dough is gently stretched and then folded over itself. This process is
typically repeated multiple times throughout bulk fermentation.

\textbf{Scalding:} A method where boiling water is poured over flour and then cooled
down to room temperature. This process gelatinizes the starches in the flour,
resulting in a dough that retains moisture better and has an improved shelf life.

\textbf{Scoring:} Cutting the surface of the bread dough before it's baked. This
allows the dough to expand freely in the oven, preventing it from bursting in
unpredictable ways. It also provides a controlled aesthetic to the finished loaf.

\textbf{Sift:} To pass flour or another dry ingredient through a sieve to remove lumps
and aerate it.

\textbf{Soaker:} A mixture of grain and water that is left to soak overnight (or for a
specified amount of time) before being incorporated into bread dough. This helps to
soften and hydrate the grains, making them easier to integrate into the dough and
providing a moister crumb in the finished bread.

\textbf{Sponge:} A type of preferment, a sponge is a wet mixture of flour, water, and
yeast that is allowed to ferment for a certain period before being incorporated into
the final dough.

\textbf{Starter:} A fermented mixture of flour and water containing a colony of
microorganisms including wild yeast and lactic acid bacteria. It's used to leaven
bread.

\textbf{Tangzhong:} A Chinese technique for bread-making, similar to the
Japanese \textbf{\#Yudane} method. It involves cooking a small portion of the flour
with water (or milk) to create a slurry or roux. This process, which can be seen as a
variant of \textbf{\#Scald}, gelatinizes the starches in the flour, resulting in breads
that are softer, fluffier, and have improved moisture retention. Once cooled, the
Tangzhong is mixed with the remaining ingredients to produce the final dough.

\textbf{Tight Crumb:} Refers to a bread crumb (the soft inner part of the bread) that
has small, uniform air holes.

\textbf{Wild Yeast:} Naturally occurring yeast, present in the environment and on the
surface of grains, used in sourdough fermentation as opposed to commercial yeast.
There’s wild yeast on almost any surface of plants. The wild yeasts live in symbiosis
with the plant providing a shield against pathogens and receiving sugars from the
photosynthesis of the plant in return. When the plant becomes weak the wild yeasts
can become parasitic and consume the host.

\textbf{Yeast:} Microorganisms that ferment the sugars present in the dough, producing
carbon dioxide and alcohol and thereby causing the dough to rise.

\textbf{Yudane:} A Japanese method of bread-making which involves the preparation
of a starter by mixing boiling water with bread flour in a specific ratio, typically 1:1
by weight. After mixing, the paste is left to cool to room temperature and then
refrigerated overnight. The next day, it is combined with the remaining ingredients
to make the dough. The Yudane method, essentially a type of \textbf{\#Scald}, helps in
improving the texture of the bread, making it softer and fluffier while also enhancing
its shelf life.
