If there is the one food from Germany it is probably bread. There are thousands
of different varieties of bread in Germany. Making bread has been an integral part
of our culture.

My bread journey began already a earlier in my childhood. My mother being parent
of 3 would always use saturdays to bake a delicious loaf of bread for the family.
It was a white fluffy sandwich bread made within 1-2 hours using yeast.
Being a bit more experienced now I know that it's
ideal to wait a bit before cutting slices of your bread. But back then
we kids couldn't help it. Mom had to directly cut us a few slices. We would then
directly proceed and pour butter or jam on each slice. Within a few minutes 1kg of
flour has been consumed. Bread became an integral part of my weekly food.

I was lucky that my parents were able to afford a yearly skiing trip to
Alto Adige in northern Italy. In the small town called Valdaora we
would try new restaurants every year, but ultimately always end up in our favorite
pizza place. The pizzas we had were incredibly delicious. The dough
alone was so tasty that ordered blank pizzas with purely the dough.
Of course my question was always, mom - can we make this at home too please?
So over the years as we became friends with the owners we would receive
more and more clues as how to make the perfect pizza dough. There
are no secret ingredients inside. It's just flour, water, salt and a bit of yeast.
How can such a simple combination of ingredients create such an incredibly delicious
pizza dough? My parents being persons of habit would return every year with us.
Each year my interest grew. At home mom and me would try to replicate
the recipe. We tried baking on a stone, a steel, adding oil to the dough,
adding herbs to the pizza sauce. We ended up in a never ending cycle
of experiments. However, we never managed to get close to the experience
that we had on vacation.

A few years later I began my studies in the small German city Göttingen.
For the first time I was faced with shopping for my own bread. It was never
on my mind to actually start baking my own bread. I would just buy me
a good loaf of bread while shopping in the supermarket. My favorite variety
used to be a Schwarzbrot, Korn an Korn. It's a very dark hearty rye bread,
consisting of lot of rye berries and sunflower seeds. Being a little naive
I never before looked at the packaging of what I was actually buying. That
changed one day. I looked at the packaging and was shocked. The seemingly
healthy bread consisted of all sorts of things other than flour and water.
The black color was not coming from the flour, but from caramelized sugar.
The packaging stated it was a sourdough bread. Why was there additional yeast
added to the process though? I thought sourdough doesn't require additional
yeast? I realized that something is rotten in the state of Denmark.
Quickly I proceeded to check all the other supermarket breads, only to
notice that all of them contained ingredients I never heard of. I lost trust
in all supermarket bread.

At home I decided to read on how to make bread. Much to my surprise I learned
that the recipe for making a pizza and bread are quite similar. Some recipes
would call for fresh yeast, others would call for dry yeast. Deep diving
into some forums I read lengthy discussions and was even more confused.
I tried to use different flours from different brands, organic flours,
non organic flours. I realized I know nothing about making bread. Recipes
would very often confuse me, because they contradicted each other. The recipes
where just a collection of seemingly random steps to follow. The baking instructions
and temperatures were all different too.

I have always been a very lazy student. I would try to reach the bare minimum
grade in order to pass the exams. I didn't bother so much. Setting low goals
allowed me to be very relaxed during the exam and keep a cool head. At the
same time I was already busy working on my own startup ideas. I hoped that they would change
the world and primarily make me rich. I was successful in a way that I would become
richer with many painful experiences and sleepless nights. I wasted my early
20s on working day and night. The wished money did not follow. Instead
I learned how it is to feel completely broke and pile up debts. Being
an entrepreneur can be a very stressful experience. For every success story,
there are probably a hundred more of failure. Failure didn't bother me so much though.
Now that I am thinking about it, it had probably always been the invisible
safety net my parents were able to provide me with. At one time I was
severely broke and then received an additional 10k invoice from my health insurance.
My parents supported me with a loan. Being broke allowed me to realize
to see how privileged I actually was. It changes your perspective on many things.
In that way I am happy now that I had to go through this painful experience and crisis.

Having completed my studies I started to work as an engineer.
When working as an engineer you are faced with challenges. The compiler or runtime
always screams at you with errors you have to fix. Sometimes it can take hours or
even days to fix a simple problem. If you want to become a software engineer
you have to develop a certain never giving up mindset. Frequently when writing code,
a set of pre-made routines is used by developers. These routines have been
programmed before by other engineers and can then be used to ship code faster.
The pre-written code is commonly known as {\it a framework}. In many cases
these frameworks are not built by a single person. The source code is published
and other engineers from all around the world can help improving and changing
the source code. Frameworks have made many successful businesses possible. Working
with frameworks in most cases they do exactly what they say they do. However
sometimes you are faced with issues you don't understand. In 99.95 percent
of all bugs in software, the developer writing the code is probably the issue.
Sometimes though the framework has a bug. This is when you have to dig deeper and
see what and why the framework is doing something. You will need to read other
engineer's source code. You are forced to understand why things are happening.

Being not happy with what I was baking, my engineering mindset took over. I had
to deep dive and understand exactly what happened. Much to my surprise none
of the recipes I could find would tell me why I should use exactly amount X
of water for flour Y. Why exactly should I use fresh yeast over dry yeast?
Why should I slap my dough while kneading on the counter? Why is a standmixer
better than kneading by hand?  Why should I let the dough sit for this long?
Why is steaming the dough during baking important? Do I really need to
get myself an expensive dutch oven to bake bread? This became even worse
when I started to read about sourdough. It sounded like dark magic to me.
Some sourdoughs were made from fruits, other from flour? Why do some
people have a wheat, a rye and a spelt sourdough? How often should the sourdough be
fed? All my questions back then could probably fill 20 pages. I was confused
but became more and more determined to find out how decent bread at home
should be made.

The feedback from my friends had improved and improved with each
iteration of homemade bread. Compared to coding where
you sometimes have to wait for months to receive feedback on what you do,
bread making is much more direct. Plus you can eat your successes and failures.
Much to my surprise even my failures would start to taste
better than most of the store-bought breads. Eating a homemade bread that
took you hours to make allows you to develop a different relationship
with your food. Making a bread from scratch with my bare hands
was a much welcome change after hours of working on the computer.
I would keep learning more about the process of fermentation and
various techniques of bread making. I approached the topic of sourdough bread
in a very similar way how I approach software.

After several years of learning and documenting my progress I decided it was
time to share my ideas with the world. When working on open source projects
it is important to see a history on how the code changes over time. This
way you can jump back to previous versions of the source code. This was
the perfect tool for documenting my recipes, because my recipes would also change
with each subsequent iteration. Much to my surprise my open source sourdough
work had been appreciated by many other engineers. The project became
very popular on the website GitHub, originally built to share software
source code. When baking great bread you also need to learn certain techniques.
I figured it was easier to share these techniques as a video. That's how
my YouTube channel was created. I chose the name {\it The Bread Code} to
visualize my software engineering approach to bread. The channel only gained
viewers when I started to choose more engaging thumbnails and titles for
the videos that I made. 3 years later I now reserve 2 days per week
to follow my bread baking passion. 3 days per week are used for my engineering
job where I still write code on a daily basis. My bread days continue
to fill me with a lot of joy and passion. To me there is nothing better than
seeing how people make amazing bread thanks to my tips and explanations.
The community has grown and grown, providing many interesting discussions
and ideas surrounding the topic of bread making. There is always something
new to learn and I feel that even now I am just barely scratching the surface with
what I know and teach. Would you ever have imagined that fruit flies are like
bees and are part of the wild yeast's success story? I made a video where
I tried to cultivate wild yeast spores coming from fruit flies in order
to bake a bread. It worked, the bread turned out amazing and even tasted
very good. These kind of experiments spark my natural interest. Conducting them
and seeing how other people share my interest makes me very happy.

The problem with running a YouTube channel is that all the information
you see is filtered and provided to you through an algorithm. I am very
worried how algorithms are shaping modern information. They tend to
put users into certain categories where you only see news related
to the bucket that you have been placed in. A key metric is how many
people click on a video after it has been shown. The content you create
is not even shown to every subscriber of your channel. If the algorithm
determines the video is not engaging enough, your content starts to
decay in YouTube's nirvana. If your video goes viral, the algorithm
will stop showing it once engagement rates with new users go down.
Older videos are slowly fading over time, as the decay punishment
factor increases and increases. I have been developing similar algorithms
myself as a software engineer.

I decided I want to try taking some time off from the algorithm cycle and
work on something more long term, something even more meaningful.
My mission has always been to share my knowledge with as many people
on the world as possible. That's also why my content has been provided
in English and not in German. After discussing with members of the community
I figured that writing a book could help me to achieve that goal. Most
current books are collections of recipes. My idea was to provide you
with a deeper solid knowledge foundation that you can use to follow other recipes.
In software terms a {\it bread framework}. This book could help everyone
facing issues with flour, fermentation, baking and much more. It would provide
a detailed understanding on why certain steps should be done and how to
adapt when things go south while making bread.  However
I want my knowledge to be accessible by everyone around the world. I do
not want to charge for the book. I want the information to be freely accessible,
no matter your budget. That's why I decided to make the book open source
and ask the community to support funding my project via my ko-fi page
(https://ko-fi.com/thebreadcode).
The feedback from the community has been amazing and I already raised a lot more money than
I initially expected. The first version of the book will only be available
digitally. This way everyone can read the book. There might be a hardcover
version too in the future, depending on how well this book is appreciated
by bakers around the world. The hardcover version will cost a bit of money,
while the digital version will always be completely free.

With this book I will try to be as scientific as possible, but in no way
claim that this is a work of science. I have conducted several experiments
myself that I will show you in this book. But to call this science, you would
probably need to repeat the same experiment a thousand times in a lab
environment. I will do my best to provide scientific references where possible
and clearly distinguish between facts and my personal opinion.

I hope you have fun reading this book and that you learn a lore more about
the fascinating world of bread making. I sincerely wish that this work
will provide you with a solid toolchain that I wish I had when starting
to make bread.

Thank you.
Hendrik