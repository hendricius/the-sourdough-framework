\chapter{Mix-ins}%
\label{ch:mix-ins}
\begin{quoting}
    This work-in-progress chapter will describes altering and additions you
    could make to your dough to create beautiful or different tasting loafs.
\end{quoting}

A loaf of wheat sourdough has a very pure aesthetic. Good craftsmanship and
precision transforms the ingredients into simple, but delicious food. With
mix-ins, the basic recipe can become the starting point for a whole world of
modifications to try and combine. Think of the loaf of bread as a blank canvas
to express yourself.

One approach to sort through the options is to categorize mix-ins by shape
(the transition between these categories is somewhat fuzzy):
\begin{itemize}
  \item Liquids: Integrate homogeneously into the dough, may replace some of
      the water. Examples: Milk, oil, spinach juice.
  \item Powders: Integrate homogeneously into the dough, may replace some of
      the flour. Examples: Rye flour, semolina, cocoa, ground spices.
  \item Small bits: Individually visible in the final loaf, small enough to
      distribute somewhat evenly throughout the dough. Examples: Seeds (poppy
      seeds, sesame, pumpkin seeds), whole spices (coriander).
  \item Chunks: Larger pieces that will only be present in the occasional bite
      when eating a slice of your bread. Examples: dried tomatoes, chunks of
      cheese,
\end{itemize}

Another categorization approach looks at the changes to the bread. Most
mix-ins actually impact multiple aspects.
\begin{itemize}
  \item Flavor: Significantly changes the taste of the bread. Examples: rye
      flour, spices.
  \item Color: Significantly changes the look of the bread. Examples: cocoa,
      squid ink, beetroot juice.
  \item Texture: Significantly changes the feeling in the mouth when eaten.
      Examples: Cheese (gummy), seeds (crunchy), olives (squishy chunks).
\end{itemize}

Mix-ins affect the structure of the dough. One aspect is the impact on
hydration. Some mix-ins absorb a lot of water when added to the dough, so you
have to increase the amount of water to achieve the same dough consistency.
The other impact is on the gluten network. Bits and chunks disrupt the gluten
network, and may reduce the rise. All of this depends on the amount of mix-ins
used. A good rule of thumb is to add \qtyrange{10}{20}{\percent} of the amount
of flour in most mix-ins, reduced to around \qtyrange{1}{5}{\percent} of the
amount of flour for spices.

An important factor is also the mix-in's behavior during baking. Particularly
chunks may bake differently than dough, and either melt (cheese) leaving holes
inside, or char when peeking through the crust (\eg, vegetables). These
problems can be mitigated to some degree with the right preparation (\eg,
chopping into smaller pieces, soaking dry ingredients in water or oil first,
or squeezing out excess moisture).
% potential reference to link: https://food52.com/blog/25521-additions-to-sourdough-bread-ideas

\section{Examples}

The following is a list of common mix-ins and their peculiarities:

\subsection{Flours}
These are powders. Usually you want to just replace some fraction of the
regular bread flour. Different flours change the taste of the bread and
usually moderately affect the color.
\begin{itemize}
  \item Whole wheat flour (substitute any amount, makes the bread taste more
      complex, nutty)
  \item Rye flour (very hearty, nutty, malty taste)
  \item Semolina (supports mediterranean flavors)
  \item Cocoa (replace \qty{10}{\percent} of the flour for a black loaf, goes
      great with sweet toppings)
\end{itemize}

\subsection{Liquids}
Substitute some of the water with a different liquid, affecting taste and
texture.
\begin{itemize}
  \item Coffee
  \item Beer
  \item Olive oil (mediterranean)
  \item Milk (for sweet, soft breads)
  \item Buttermilk
\end{itemize}

\subsection{Colors}
These drastically change the color of the bread.
\begin{itemize}
  \item Beetroot juice (red)
  \item Carrot juice (orange)
  \item Spinach juice (green)
  \item Squid ink (black)
\end{itemize}

\subsection{Seeds and nuts}
These are small bits, with some almost crossing into the chunk category. Most
seeds benefit from being baked for about 10~minutes before adding them to the
dough.
\begin{itemize}
  \item Pumpkin seed
  \item Chia seed
  \item Flaxseed  (soak these in water first)
  \item Hemp seed (very crunchy, a personal favorite)
  \item Sesame
  \item Sunflower seed
  \item Poppy seed
  \item Cacao nibs
  \item Chopped or whole walnuts
  \item Chopped or whole hazelnuts
\end{itemize}

\subsection{Spices and flavor mix-ins}
These are mostly powders or small bits.
\begin{itemize}
  \item Mediterranean herbs (oregano, thyme, rosemary, marjoram)
  \item Bread spice (coriander, cumin, fennel, anise)
  \item Grated hard cheese: Gruyère, parmesan
  \item Blueberry skins (press through sieve to remove juice, raw blueberries
      would add too much water)
  \item Lemon zest (alternatively orange or lime)
  \item Browned onions
  \item Molasses
  \item Miso
\end{itemize}

\subsection{Highlights}
Mostly chunks, that add a big contrast and flavorful highlight to the basic
bread. Usually you want to use only one (or maximum two) of these. Often can
be complemented well by some flavor mix-in or flour.
\begin{itemize}
  \item Olives
  \item Sundried tomatoes (squeeze out the oil if using pickled ones, or soak
      dried ones in water)
  \item Pickled pepperoni
  \item Cornflakes
  \item Dried fruit (\eg, cranberries, raisins)
  \item Chunks of cheese (\eg, cheddar, feta)
  \item Chunks of black garlic
  \item Chocolate chunks or drops
\end{itemize}

\subsection{Combinations}
A few combinations where multiple mix-ins complement each other:
\begin{itemize}
  \item Semolina, mediterranen herbs, olives, sundried tomatoes.
  \item Cranberry and walnuts.
  \item Cheddar and pepperoni.
  \item Cocoa, cacao nibs, whole hazelnuts.
\end{itemize}

\section{Techniques}
Adding mix-ins into the dough is just the simplest approach. There are other,
more advanced ways to include them into a loaf.

\subsection{Covering the crust}
This works best for either powders or small bits. Spread the mix-in in a flat
container, wet the surface of the loaf, and dip it into the mix-in right
before baking.

This does not work for all mix-ins, as some can't handle the high temperatures
during baking and char. Most commonly done with seeds (\eg, sesame).

\subsection{Swirled colors}
Mix-ins that change the color of the dough bring the opportunity for even more
creativity.

Separate the dough before adding a colorful ingredient. Combine the two (or
more) differently colored doughs by laminating and stacking the colored sheets
of dough before the last folding, just before shaping and bulk rise.

These can really become works of art.

% https://www.reddit.com/r/Sourdough/comments/onynqm/sourdough_with_dried_raspberries_recipe_in/
% https://natashasbaking.com/blueberry-sourdough/
% https://www.reddit.com/r/Sourdough/comments/mot8vq/chocolate_sourdough_loaf/
% https://www.reddit.com/r/Sourdough/comments/13sdex9/fairy_bread_for_my_daughters_class_party_with/
% https://www.reddit.com/r/Sourdough/comments/keyx88/roasted_onion_and_garlic_loaf_this_loaf_didnt/
% https://myloveofbaking.com/rye-molasses-and-orange-sourdough/
% https://www.reddit.com/r/Sourdough/comments/qd3y4k/pick_your_player_miso_sesame_or_cranberry_walnut/
% https://www.reddit.com/r/Sourdough/comments/lziedg/10_spelt_flour_80_hydration_50_buttermilk_50/
% https://www.reddit.com/r/Sourdough/comments/lbrc4a/squid_ink_sourdough_with_sharp_cheddar_and/
% https://www.reddit.com/r/Sourdough/comments/na0zed/was_hoping_for_a_more_pronounced_purple_but_i/
% https://www.reddit.com/r/Sourdough/comments/10rzgif/sesame_and_poppyseed_64_hydration/
% https://www.reddit.com/r/Sourdough/comments/11lcgvr/sesame_seed_crusted_loaf_w_everything_bagel/
