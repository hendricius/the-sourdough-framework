\subsubsection*{Ingredients}
\begin{tabular}{r@{}rl@{}}
\qty{400}{\gram} &~(\qty{100}{\percent}) & Flour (wheat, rye, corn, whatever
                                            you have at hand)\\
\qty{320}{\gram} &  (\qty{80}{\percent}) & Water, preferably at room
                                            temperature\\
\qty{80}{\gram}  &  (\qty{20}{\percent}) & Active sourdough starter\\
\qty{8}{\gram}   &   (\qty{2}{\percent}) & Salt\\
\end{tabular}

\subsubsection*{Instructions}
\begin{description}
\item[Prepare the dough] In a large mixing bowl, combine the flour and water.
    Mix until you have a shaggy dough with no dry spots.

    Add the sourdough starter and salt to the mixture. Incorporate them
    thoroughly until you achieve a smooth and homogenized dough.

\item[Fermentation:] Cover the bowl with a lid or plastic wrap. Allow the dough
    to rest and ferment until it has increased by at least \qty{50}{\percent}
    in size.  Depending on the temperature and activity of your starter, this
    can take anywhere from 4 to 24~hours.

\item[Cooking preparation:] Once the dough has risen, heat a pan over medium
    heat.  Lightly oil the pan, ensuring to wipe away any excess oil with a
    paper towel.

\item[Shaping and cooking:] With a ladle or your hands, scoop out a portion of
    the dough and place it onto the hot pan, spreading it gently like a
    pancake.

    Cover the pan with a lid. This traps the steam and ensures even cooking
    from the top, allowing for easier flipping later.

    After about 5~minutes, or when the bottom of the flatbread has a
    golden-brown crust, carefully flip it using a spatula.

    \emph{Adjusting cook time.} If the flatbread appears too dark, remember to
    reduce the cooking time slightly for the next one.  Conversely, if it's
    too pale, allow it to cook a bit longer before flipping.

    Cook the flipped side for an additional 5~minutes or until it's also
    golden brown.

\item[Storing:] Once cooked, remove the flatbread from the pan and place it on
    a kitchen towel. Wrapping the breads in the towel will help retain their
    softness and prevent them from becoming overly crisp.  Repeat the cooking
    process for the remaining dough.

\item[Serving suggestion:] Enjoy your sourdough flatbreads warm, paired with
    your favorite dips, spreads, or as a side to any meal.

\end{description}
