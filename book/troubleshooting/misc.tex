\section{Baking in the tropics}

Depending on the temperature your fermentation speed adapts.
In a warmer environment everything is faster. In a colder
environment everything is slower.

This includes the speed at which your sourdough ferments
the dough but also the speed of enzymatic reactions. The
amylase and protease enzymes work faster, making more
sugars available and degrading the gluten proteins.

At around 22°C in my kitchen my bulk fermentation is ready
after around 10 hours. I am using around 20 percent of sourdough
starter based on the flour. In summer times the temperatures
in my kitchen sometimes increase to 25°C. In that case
I am reducing the sourdough starter to around 10 percent.
If I wouldn't do that my fermentation would be done after
around 4-7 hours. The problem is that the dough is quite
unstable when fermenting at this high speed. This means
that you are easily running into issues of overfermentation.
Finding the perfect sweet spot between fermenting enough
and not too much is becoming much harder. Normally you might
have a time window of 1 hour. But at the rapid speed it
might be reduced to a time window of 20 minutes. Now at
30°C ambient temperature things are way faster. Your bulk
fermentation might be complete in 2-4 hours when using
10-20 percent starter. Proofing your dough in the fridge
becomes almost impossible. As your dough cools down in the
fridge the fermentation also slows down. However cooling the
dough down from 30°C to 4-6°C in your fridge takes much
longer. Your dough is much more active compared to a dough
that starts at a temperature of 20-25°C. You might
end up overproofing your dough if you leave it overnight
in the fridge.

That's why I recommend that you reduce the amount of starter
that you use in the tropics to something at around 1-5 percent
based on the flour. This will slow down the fermentation
process significantly and provides you a bigger window
of time. Try to aim for an overall bulk fermentation of at
least 8-10 hours. Reduce the amount of starter to get there.

When making a dough try to use the same water temperature
as your ambient temperature. Assuming that the temperature
will climb to 30°C, try to start your dough directly
with 30°C water. This means that you can carefully rely on
a small fermentation probe that visualizes your fermentation
progress. The probe only works reliably if your dough temperature
is equal to your ambient temperature. Else the sample heats
up or cools down faster. So tread carefully when using
the sample in this case. It's always better to stop
the fermentation a little too early rather than too late.
Stretch and folds during the bulk fermentation
will help you to develop a better look and feel for
the dough. An expensive but possibly useful tool
could be a pH meter that allows you to perfectly
measure how much acidity has been created by the
lactic and acetic acid bacteria. In this case measure
the pH repeatedly and figure out a value that works
for your sourdough. In my case I tend to end bulk
fermentation at a pH of around 4.1. Please don't just
follow my pH value, it's very individual. Keep measuring
with different doughs to find out a value that works for you.

\section{My bread stays flat}
\section{I want more tang in my bread}
\section{My bread is too sour}
\section{Fixing a moldy sourdough starter}
\section{My bread flattens out after shaping}
\section{Liquid on top of my starter}
\section{Why does my starter smell like acetone}
\section{My crust becomes chewy}